\section{Sets and Proofs}

\subsection{Sets}

A \textit{set} is a collection of objects called \textit{elements} or \textit{members}.
In simple cases, we can enumerate all members with this notation: $\{1, 3, 5\}$.
Often we cannot list all of the elements.
For example, "the set of all real numbers whose square is less than 2" can be expressed as

$$
\{x | x \text{ a real nnumber}, x^2 < 2\}
$$

\noindent where $|$ is read as "such that."

The \textit{empty set}, $\emptyset$, is a set with no objects.

If $s$ is in set $S$, we say "$s$ belongs to $S$" and denote it as $s \in S$.
Conversely, if $s$ is not in $S$, it is denoted as $s \notin S$.

Two sets are \textit{equal} when they contain exactly the same elements.
$T$ is a subset of $S$ if all elements of $T$ also belong to $S$; this relationship is denoted as $T \subseteq S$, or $T \nsubseteq S$ if $T$ is not a subset of $S$.
The empty set is a subset of all sets.

\subsection{Proofs}
