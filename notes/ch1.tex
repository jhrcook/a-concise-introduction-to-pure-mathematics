\section{Sets and Proofs}

\subsection{Sets}

A \textit{set} is a collection of objects called \textit{elements} or \textit{members}.
In simple cases, we can enumerate all members with this notation: $\{1, 3, 5\}$.
Often we cannot list all of the elements.
For example, "the set of all real numbers whose square is less than 2" can be expressed as

$$
\{x | x \text{ a real number}, x^2 < 2\}
$$

\noindent where $|$ is read as ``such that.''

The \textit{empty set}, $\emptyset$, is a set with no objects.

If $s$ is in set $S$, we say ``$s$ belongs to $S$'' and denote it as $s \in S$.
Conversely, if $s$ is not in $S$, it is denoted as $s \notin S$.

Two sets are \textit{equal} when they contain exactly the same elements.
$T$ is a subset of $S$ if all elements of $T$ also belong to $S$; this relationship is denoted as $T \subseteq S$, or $T \nsubseteq S$ if $T$ is not a subset of $S$.
The empty set is a subset of all sets.

\subsection{Proofs}

The purpose of a \textbf{proof} is to provide a convincing, logical justification of an opinion.

The notation $P \implies Q$ translates to ``statement $P$ implies statement $Q$.''
It can also be read as \begin{itemize}
    \item ``if $P$, then $Q$''
    \item ``$Q$ if $P$''
    \item ``$P$ only if $Q$''
\end{itemize}
\noindent{}Note that $P \implies Q$ does not necessitate that $Q \implies P$.
If this were the case, use the notation $P \iff Q$ which translates to ``$P$ if and only if $Q$''.

The notation for negation, e.g. ``not $P$,'' is $\bar{P}$.

With this notation, the following is logically true: if $P \implies Q$, then $\bar{Q} \implies \bar{P}$.

\subsubsection*{Example 1.2}

Proving the following statement: ``The square of an odd integer is odd.''

\begin{proof}
    Let $n$ be an odd integer.
    Then $n$ is 1 more than an even integer, so $n = 1 + 2m$ for some integer $m$ (odd or event).
    Then $n^2 = (1+2m)^2 = 1 + 4m +4m^2 = 1 + 4(m + m^2)$.
    $4(m + m^2)$ is even, therefore that $+1$ is odd, showing $n^2$ is odd.
\end{proof}

This could have been written as:

$$
n \text{ odd} \implies n = 1 + 2m\ \forall\ m \in \mathbb{N} \implies n^2 = 1 + 4(m+m^2) \implies n^2 \text{ odd}
$$

\noindent{}but it is more difficult to read.
Therefore, it is common (encouraged?) to include explanatory English text in a proof.

This is an example of a \textit{direct proof} because it proceeds from assumptions to conclusions via a series of implications.

%  TODO: on the section of proof by contradiction
